\section{Simulation of CM Chokes at a System Level} \label{sec:simulation}

\subsection{Multi conductor transmission line (MTL) solver} \label{sec:mtl-solver}
The MTL method assumes a quasi-TEM propagation applicable in systems that can be decomposed on groups of conductors with constant cross-sections, i.e., translational symmetry.
This assumption permits us to relate the voltages and currents on each conductor with per-unit-length parameters \cite{Paul2007}.
For a lossless line with $(N+1)$ conductors, these relationships can be expressed in the following form
\begin{align}
	\partial_z \voltageVec &= -\Rpul \currentVec  - \Lpul \, \partial_t \currentVec \nonumber \\
	\partial_z \currentVec &= -\Gpul \voltageVec  - \Cpul \, \partial_t \voltageVec \label{eq:mtl}
\end{align} 
with $\voltageVec(z,t) = \left[ V_1 \, V_2 \, \hdots V_N \right]^T$ being a vector of the voltages with respect to a conductor taken as ground and $\currentVec (z,t) = \left[ I_1 \, I_2 \, \hdots I_N \right]^T$ is a vector of currents on each of these conductors. 
The matrices $\Lpul$ and $\Cpul$ represent the inductance and capacitance per unit length between all conductors.
$\Rpul$ and $\Gpul$ represent the losses in the transmission line: resistance (along conductors) and conductance (between conductors) per unit length, respectively.
Finally, \autoref{eq:mtl} is discretized using the finite-difference time-domain technique (FDTD) described in \cite{Paul2007}, leading to the following voltage and current update equations:

{\small
	\begin{align}
		\voltageVec^{n+1}_{k} &= \left(\frac{\Delta z}{\Delta t}\Cpul + \frac{\Delta z}{2}\Gpul \right)^{-1} \left[ \left(\frac{\Delta z}{\Delta t}\Cpul - \frac{\Delta z}{2}\Gpul \right) \voltageVec^{n}_{k} \right. \nonumber \\
		&- \left. \left(\currentVec^{n+1/2}_{k}-\currentVec^{n+1/2}_{k-1}\right) \right] \\
		\currentVec^{n+3/2}_{k} &= \left(\frac{\Delta z}{\Delta t}\Lpul + \frac{\Delta z}{2}\Rpul \right)^{-1} \left[ \left(\frac{\Delta z}{\Delta t}\Lpul - \frac{\Delta z}{2}\Rpul \right) \currentVec^{n+1/2}_{k} \right. \nonumber \\
		&- \left. \left(\voltageVec^{n+1}_{k+1}-\voltageVec^{n+1}_{k}\right) \right] \label{eq:fdtd-i-update}
	\end{align} 
}%
% \voltageVec^{}_
where $\Delta t$ and $\Delta z$ are the time and space steps of the discretization, respectively, $k$ locates the voltage nodes and current segments along the discretized MTL, and $n$ locates the current time on the discretized solution time. 
The details of the implementation can be found in \cite{opensemba-mtln}.

    
The quasi-TEM propagation assumption used by MTL allows to solve a wide variety of typical problems in EMC: crosstalk prediction, S-parameters extraction, susceptibility to radiated emissions, inclusion of non-linear models, etc.
Moreover, this allows a significant reduction in the complexity of the problem which directly translates into a significant computational efficiency.
However, the quasi-TEM approach does not account electric fields aligned with the cables and therefore can not predict radiative effects, e.g. a radiating electric dipole.
For this reason, MTL typically produces good results, mainly below the system resonant frequencies, i.e., small electric size. For this reason, MTL typically produces good results mostly below the resonant frequencies of the system, i.e. small electric size. 

The per-unit-length parameters in $\Rpul$, $\Lpul$, $\Gpul$, and $\Cpul$ can generally depend on the frequency. 
However, for our purposes, we will assume that $\Lpul$ and $\Cpul$ are constant and that no losses between conductors exist, $\Gpul = 0$. 
The behavior of the choke will appear as an impedance $\Rpul = \Zwpul(z, \omega)$ dependent on the frequency and on the position. 
That is, $\Zwpul$ is non-zero only at the position where the lumped component is located. The equation analogous to \autoref{eq:mtl} in the frequency-domain is:
\begin{equation}\label{eq:mlt-i-freq}
	\partial_z \voltageVecW = -\Zwpul \currentVecW  - \iu \omega \Lpul \currentVecW(z, \omega) 
\end{equation}

These frequency-domain relationships can be translated into the time domain convolutionally as

\begin{align}
	\partial_z \voltageVec &= -\Zpul(z,t) \ast \currentVec(z,t)  - \Lpul \, \partial_t \currentVec \nonumber \\
	\partial_z \voltageVec &= -\int_0^{t} \Zpul(z,\tau) \currentVec(z,t-\tau)  - \Lpul \, \partial_t \currentVec \label{eq:conv}
\end{align}

To express \autoref{eq:conv} in a way that can be readily discretized and implemented into the FDTD scheme, we apply the piecewise linear recursive formulation proposed in \cite{Oh1995}. In this formulation, a frequency-dependent magnitude is approximated by a rational representation (a sum of complex pole-residue pairs), obtained using the vector fitting technique \cite{Gustavsen1999,Gustavsen2006}. The inverse Fourier transform of this representation is a sum of exponentials whose convolution can be expressed recursively. 

If the frequency-dependent impedance is approximated by:
\begin{equation} \label{eq:Z-VF}
	\complexImpedance(s)\,=\,\complexCte + s\complexProp + \sum_{i} \frac{\complexResidue_i}{s-\complexPole_i} + \sum_{i} \frac{\complexResidue_i^{*}}{s-\complexPole_i^{*}}
\end{equation}
where $s\,=\,\iu\omega$, and $\complexResidue_i$ and $\complexPole_i$ are the \textit{i}th residue and pole, respectively, and its time-domain equivalent can be written as
\begin{equation} \label{eq:Z-VF-TD}
	\complexImpedance(t)\,=\,\complexCte\delta(t) + s\complexProp\delta^{\prime}(t) + \sum_{i} \complexResidue_i e^{\,\complexPole_i\cdot t}
\end{equation}
The convolutional term is then
\begin{align}
	\int_0^{t} \Zpul(z,\tau)  \currentVec(z,t-\tau) =& \complexCte \currentVec + \complexProp \partial_t \currentVec\,+ \nonumber \\ 
	=& \sum_{i} \complexResidue_i \int_0^{t} e^{\,\complexPole_i\cdot t} \currentVec(z,t-\tau) \label{eq:conv-vf}
\end{align}
If \autoref{eq:conv} is discretized, the sum term in \autoref{eq:conv-vf} takes the following form
\begin{align}
    \sum_i \complexResidue_i \int_0^{t} e^{\,\complexPole_i\cdot t} \currentVec(z,t-\tau) =& \, \sum_i  \varphi^{n}_{i}
    \label{eq:dis-con}
\end{align}
where 
\begin{align}
	\varphi^{n}_{i} &= \currentVec^{n+3/2}_{k}\,q_{1i} + \currentVec^{n+1/2}_{k}\,q_{2i} \nonumber  + \varphi^{n}_{i}\,q_{3i} \label{eq:conv-vf-2} \nonumber \\
	q_{1i} &= \frac{\alpha_i}{\beta_i}\,(e^{\beta_i}\,-\,\beta_i\,-\,1)\nonumber \\
	q_{2i} &= \frac{\alpha_i}{\beta_i}\,(1\,+\,e^{\beta_i}(\beta_i\,-\,1))\nonumber \\
	q_{3i} &= e^{\beta_{i}}\textrm{,} \hspace{10pt} \alpha_i\,=\,\frac{\complexResidue_i}{\complexPole_i}\textrm{,} \hspace{10pt} \beta_i\,=\,\complexPole_i\,\delta t\nonumber 
\end{align}
in which the convolution integral has been replaced by a recursive sum. 
Now, inserting \autoref{eq:dis-con} into \autoref{eq:conv} yields to the discretized update relation for $\currentVec$, analogous to \autoref{eq:fdtd-i-update},  
\begin{align}
	\currentVec^{n+3/2}_{k} &= F_{1}^{-1}\left[F_{2}\currentVec^{n+1/2}_{k} - (\voltageVec^{n+1}_{k+1}-\voltageVec^{n+1}_{k}) \right. \nonumber \\
	&- \left. \Delta z \sum_{i}q_{3i}\varphi^{n-1}_{i}  \right]
\end{align}
where
\begin{align}
    F_{1} &= \left( \frac{\Delta z}{\Delta t}\Lpul + \frac{\Delta z}{2}\complexCte + \frac{\Delta z}{\Delta t}\complexProp + \Delta z \sum_{i}q_{1i}\right)\\
    F_{2} &= \left( \frac{\Delta z}{\Delta t}\Lpul - \frac{\Delta z}{2}\complexCte + \frac{\Delta z}{\Delta t}\complexProp - \Delta z \sum_{i}q_{2i}\right)
\end{align}

\subsection{Full-wave FDTD solver} \label{sec:full-wave-fdtd}
The full-wave FDTD method is widely used to analyze transient fields in 3D.
The standard version of this technique is based on tessellating the space with cubes known as Yee cells, locating the electric and magnetic fields in their edges and faces \cite{Yee1966}.
This staggered arrangement of the fields is evolved with a second-order leapfrog algorithm, which results in a scheme that has the following form (in free space, for simplicity)
\begin{align}
	\{E^{n+1}  \} & = \{E^n      \} + [A_{H}] \{H^{n+1/2}\}+\{S^{n+1/2}\}  \nonumber \\
	\{H^{n+3/2}\} & = \{H^{n+1/2}\} + [A_{E}] \{E^n\} \label{eq:fdtd-base-algorithm}
\end{align}
with $\{E\}$ and $\{H\}$ being the discretized fields in the Yee's cell and $\{S\}$ the discrete electric-source terms. $[A_{H}]$ and $[A_{E}]$ are sparse evolution matrices that take into account the finite-differences spatial semi-discretization and boundary conditions.

While this scheme is suitable for a wide range of applications, it has limitations when multiple geometrical scales are involved in a single problem.
Therefore, to apply it to structures containing wires, the base algorithm can be modified with the well-known Holland method for thin wires \cite{Holland1981,Schmidt2004}.

Holland's method employs for a thin wire the same equations (\ref{eq:mtl}) used in Section \ref{sec:simulation}, now for a single TL, by assuming, roughly speaking, that the surrounding space acts as a reference {\it current return} of the wire. The scalar inductance per unit length is $L=\langle\frac{\mu_0}{2\pi}\ln\rho\rangle$, related to the capacitance by $C=1/(Lc^2)$, with $c$ the free-space light speed, and $\langle\cdot\rangle$ denoting some integral average around the thin wire (full details can be found in \cite{Holland1981}.) 

The TL single-wire Holland's equations are discretized by an FDTD scheme. They are coupled with the Yee algorithm at the wire locations and at each time step in a two-way explicit manner: they use the FDTD E-fields as distributed voltage sources in Holland equations (independent terms added to (\ref{eq:mtl}), and return the currents as electric-source terms to the FDTD equations (\ref{eq:fdtd-base-algorithm}).

Incorporating the ferrite choke in the Holland model thin wire uses the same method described in Section \ref{sec:simulation} for the MTL case: a dispersive lumped impedance at the choke location. The only difference between the MTL and the full-wave Holland-FDTD method is how the wires communicate among themselves: by inductance and capacitance matrices or by Yee's algorithm explicitly propagating the electromagnetic fields by Maxwell equations. 

However, in the solver we used for this paper, we have implemented a slightly different algorithm based in \cite{Antonini2003} to simulate dispersive elements in terms of their admittance instead of their impedance. For this, a poles-residues fit to the element admittance is found equivalently to \autoref{eq:Z-VF} 
\begin{equation}\label{eq:Y-VF}
	\complexAdmittance(s)\,=\,\complexCte^{\prime} + s\complexProp^{\prime} + \sum_{i} \frac{\complexResidue^{\prime}_i}{s-\complexPole^{\prime}_i} + \sum_{i} \frac{\complexResidue_i^{\prime\,*}}{s-\complexPole_i^{\prime\,*}}
\end{equation}
%
The previous relationship allows us to interpret each term easily as follows: the constant and linear frequency terms are equivalent to a parallel RC circuit; each real-pole term is equivalent to a series RL circuit, and each pair of complex-conjugate poles is equivalent to a circuit composed by a series RL in series with a parallel RC.

